\documentclass[a4paper,11pt]{article}
\usepackage{luatex85}
\usepackage[babelshorthands]{polyglossia}
\usepackage[fleqn,reqno]{amsmath}
\usepackage{amsthm}
\usepackage[math-style=ISO,bold-style=upright,partial=italic]{unicode-math}
\usepackage{microtype}
\usepackage[width=16cm,height=24cm]{geometry}
\usepackage[russian]{hyperref}

\setmainlanguage{russian}

\setmainfont{Cambria}
\setsansfont{Calibri}
\setmonofont{Source Code Pro}[Scale=MatchLowercase]
\setmonofont{Source Code Pro}[Scale=MatchLowercase]
\setmathfont{Cambria Math}[sans-style=literal]
\setmathfont{XITS Math}[range={\mathscr}]

\makeatletter

\allowdisplaybreaks[4]

\def\[#1\]{\begin{align*}#1\end{align*}}
\newcommand\eqtag[1]{\refstepcounter{equation}\tag{\theequation}\label{#1}}

%\newcommand\slashfrac[2]{{#1\fracslash#2}}
\newcommand\slashfrac[2]{{#1/#2}}

\newcommand\pr{\operatorname{\textbf{\textup{pr}}}}

\theoremstyle{definition}
\newtheorem{theorem}{Теорема}
\newtheorem*{theorem*}{Теорема}
\newtheorem{lemma}{Лемма}
\newtheorem{definition}{Определение}
\newtheorem{example}{Пример}
%\newtheorem{proof}{Доказательство}

\newcommand\metasetup{\hypersetup{
	pdftitle=\@title,
	pdfauthor=\@author,
	linkbordercolor={0 .5 .25},
	}}

\setlength\overfullrule{5pt}

\makeatother

\begin{document}

\hyphenation{Ла-гран-жа ла-гран-же-вой}

\title{Хаос в~гамильтоновых системах}
\author{А.~Н.~Швец}

\metasetup
\maketitle

Как мы знаем, фазовый поток гамильтоновой системы за время $\theta$
осуществляет каноническое преобразование фазового пространства. Многие наши
задачи были посвящены интегрируемым лагранжевым/гамильтоновым системам,
обладающим $2n$ первыми интегралами ($n$~— число степеней свободы),
определёнными почти всюду в~расширенном фазовом пространстве. Фазовые кривые
таких систем лежат в~пересчении постоянных уровней первых интегралов, что
обеспечивает их регулярное и~скучное, предсказуемое поведение.

Но регулярная динамика в~гамильтоновых системах является скорее исключением,
нежели правилом. Существование достаточного количества первых интегралов
гарантируется, согласно теореме о~выпрямлении векторного поля, лишь локально,
и~то лишь в~окрестности неособой точки поля.

Хорошее представление о~хаосе в~гамильтоновых системах даёт популярный
модельный пример, придуманный физиком Б.~В.~Чириковым в~качестве упрощённой
модели явлений, происходящих в~кольцевых ускорителях элементарных частиц.

Рассмотрим преобразование тора $(q\bmod2\pi,p\bmod2\pi)$, заданное формулой
	\[
	(q,p)\mapsto(q+p+\varepsilon\sin q,p+\varepsilon\sin q).
	\]
Его называют \emph{стандартным отображением Чирикова.} Как легко убедиться,
отображение является каноническим при любом значении параметра~$\varepsilon$
(убедитесь!). Если~бы существовал первый интеграл данного преобразования,
то~есть функция $(q,p)$, постоянная на орбитах преобразования, фазовое
пространство оказалось~бы расслоенным на инвариантные кривые, несущие орбиты.
Сами орбиты были~бы при этом либо периодическими, либо
\emph{условно"=периодическими,} либо асимптотическими к~указанным. В~первом
случае инвариантные кривые были~бы составлены из разнообразных периодических
орбит, во~втором орбиты заполняли~бы кривые всюду плотно, а в~третьем
служили~бы сепаратрисами. Так или иначе, движение в~целом оказалось~бы вполне
регулярным и~предсказуемым благодаря первому интегралу.

Именно так обстоят дела при $\varepsilon=0$. Первым интегралом служит $p$,
а~фазовый тор расслоен на кривые $p=\symup{const}$. Однако всё меняется при
положительных~$\varepsilon$.

Наше демонстрационное
\href{http://mech.math.msu.su/~shvetz/practicum/SCM.svg}{приложение} показывает
построение орбиты одной"=единственной начальной точки, выбранной более"=менее
случайно. Самые свежие 10000~точек орбиты показаны красным для удобства
наблюдения, $\varepsilon=1$. Наберитесь терпения, подождите минут~20. Хорошо
видно, что орбита заполняет густую сеть \emph{(стохастическую паутину)},
пронизанную многочисленными дырками~— \emph{островами устойчивости.} Орбиты
с~начальными условиями внутри дырок (их не видно)~— это как раз уцелевшие
периодические или условно"=периодические. Острова пока ещё демонстрируют
регулярную динамику.

Что-то похожее на отображение Чирикова происходит на фазовой плоскости
математического маятника, точка подвеса которого совершает вертикальные
гармонические колебания: в~этом случае берётся отображение фазовым потоком за
период колебаний точки подвеса. Детальное описание происходящих процессов
имеется в~учебнике четырёх авторов, раздел~15.3.3.

Другая точка зрения на описанное явление показана на нашем медитативном
\href{http://mech.math.msu.su/~shvetz/practicum/SCM.mp4}{видео}. При равномерно
меняющемся $\varepsilon$ от~$0$ до~$4$ оттенками серого подсвечены точки
фазового тора в~зависимости от «хаотичности» соответствующих орбит~— более
хаотические темнее. Это хорошая иллюстрация всеобщего торжества хаоса
и~разрушения; скоро мы всё это увидим IRL. Картина лишний раз подтверждает
тезис В.~И.~Арнольда о~хрупкости и~редкости всего хорошего и~обыденности всего
плохого.

«Хаотичность» орбит при создании видео измерялась следующим образом. Она
пропорциональна положительному времени~$n$, в~течение которого точка орбиты
$T^n(q,p)$ сближается с~точкой $T^{2n}(q,p)$ той~же орбиты на малое
расстояние~$\rho$ ($T$~— отображение Чирикова). В~основе этого подхода
к~определению хаотичности орбит лежат идеи, связанные с~гениальным в~своей
простоте \emph{алгоритмом Флойда,} называемым также \emph{алгоритмом черепахи
и~зайца.} Читателям, интересующимся возникающими алгоритмическими вопросами,
предлагаем наш
\href{http://mech.math.msu.su/~shvetz/54/inf/perl-problems/chDecimalFraction_sIdeas.xhtml#chDecimalFraction_sIdeas_sTortoiseAndHareMethod}{текст},
написанный по другому поводу и~для другой аудитории. Применение того~же подхода
к~визуализации другой очень известной динамической системы (не гамильтоновой)
описано ещё в~одном
\href{http://mech.math.msu.su/~shvetz/54/inf/perl-problems/chMandelbrotSet.xhtml}{тексте}.

\bigskip

Спасибо за внимание. \emph{Вы держитесь здесь, вам всего доброго, хорошего настроения
и~здоровья!}

\end{document}
