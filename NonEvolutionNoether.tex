\documentclass[a4paper,11pt]{article}
\usepackage{luatex85}
\usepackage[babelshorthands]{polyglossia}
\usepackage[fleqn,reqno]{amsmath}
\usepackage{amsthm}
\usepackage[math-style=ISO,bold-style=upright,partial=italic]{unicode-math}
\usepackage{microtype}
\usepackage[width=16cm,height=24cm]{geometry}
\usepackage[russian]{hyperref}

\setmainlanguage{russian}

\setmainfont{Cambria}
\setsansfont{Calibri}
\setmonofont{Source Code Pro}[Scale=MatchLowercase]
\setmonofont{Source Code Pro}[Scale=MatchLowercase]
\setmathfont{Cambria Math}[sans-style=literal]
\setmathfont{XITS Math}[range={\mathscr}]

\makeatletter

\allowdisplaybreaks[4]

\def\[#1\]{\begin{align*}#1\end{align*}}
\newcommand\eqtag[1]{\refstepcounter{equation}\tag{\theequation}\label{#1}}

%\newcommand\slashfrac[2]{{#1\fracslash#2}}
\newcommand\slashfrac[2]{{#1/#2}}

\newcommand\pr{\operatorname{\textbf{\textup{pr}}}}

\theoremstyle{definition}
\newtheorem{theorem}{Теорема}
\newtheorem*{theorem*}{Теорема}
\newtheorem{lemma}{Лемма}
\newtheorem{definition}{Определение}
\newtheorem{example}{Пример}
%\newtheorem{proof}{Доказательство}

\newcommand\metasetup{\hypersetup{
	pdftitle=\@title,
	pdfauthor=\@author,
	linkbordercolor={0 .5 .25},
	}}

\setlength\overfullrule{5pt}

\makeatother
\usepackage{enumerate}
%\usepackage[amsmath,amsthm,thmmarks]{ntheorem}
%\usepackage{ulem}

\begin{document}

\hyphenation{Ла-гран-жа ла-гран-же-вой}

\title{Приключения Теоремы Нётер в дополнительных задачах}
\author{М.~С.~Терехов}

%\metasetup
\maketitle

\section{Небольшое обобщение теоремы Нётер}


Рассмотрим обобщённые векторные поля вида
	$\textbf{v}=\sum_{i=1}^n\xi_i\partial_{x_i}+\tau\partial_t$, где
функции $\xi_i$ зависят от $(t,x,\dot x)$.
Тогда, как уже было показано в работе~\cite{bib:2}
	\[
	\operatorname{\textbf{pr}}^{(1)}\textbf{v}
		=\sum_{i=1}^n\xi_i \partial_{x_i}+\tau\partial_t+\sum_{i=1}^n (\dot\xi_i-\dot\tau\dot x_i) \partial_{\dot x_i},
	\eqtag{eq:prolongation-v}
	\]
Назовём его \emph{полем обобщённой симметрии\/} лагранжевой задачи
$L[x]=\int L(t,x,\dot x)\,dt$, если найдётся такая дифференциальная
функция $A$, что
	\[
	\pr^{(1)}\textbf{v}(\textit{L})=\dot A-L\dot\tau.
	\eqtag{eq:symmetry-cond}
	\]
Под \emph{дифференциальной функцией\/} мы понимаем функцию независимой
переменной~$t$, зависимых $x_i$ и~производных вплоть до некоторого порядка.

В~оправдание названия полей обобщённых симметрий мы сошлёмся на формулу (5.53)
стр.~408 из книжки~\cite{bib:3}.


%В оправдание названия полей обобщённых симметрий заметим, что в~своём
%инфинитезимальном действии такие поля прибавляют к~лагранжиану калибровочное
%слагаемое $\varepsilon\dot A$, что означает добавление к~функционалу~$\textit{L}$
%постоянного функционала $\varepsilon\int\dot A\,dt$ (добавка зависит лишь от
%значений функций $x(t)$ на концах отрезка интегрирования, и, таким образом, не
%меняется при варьировании). Это, конечно, никак не сказывается на стационарных
%точках $\textit{L}$, следовательно, групповые преобразования переводят
%решения~$x(t)$ уравнений Эйлера~— Лагранжа в~решения.

\begin{theorem*}[A.~E.~Noether ver2.0]
Пусть $\textbf{v}=\sum_{i=1}^n\xi_i\partial_{x_i}+\tau\partial_t$~— поле обобщённой симметрии
лагранжевой задачи~$\textit{L}$ и
	\[
	I(t,x,\dot x)=\sum_{i=1}^n(\xi_i - \tau \dot x_i)\frac{\partial L}{\partial\dot x_i} - A +L\tau.
	\]
Тогда
	\[
	\dot I=-\sum_{i=1}^n(\xi_i-\tau\dot x_i)\mbfsansE_i(L).
	\eqtag{eq:dotI}
	\]
\end{theorem*}

\begin{proof}
Прежде чем приступить к доказательству, заметим, что эта теорема при $\tau=0$
даёт нам теорему из методички~\cite{bib:1}, тем самым обобщает её. Итак, прямые
вычисления показывают нам, что 
	\[
	\dot I=\dot\tau L+\tau\dot L+\sum_{i=1}^n(\dot\xi_i-\dot\tau\dot x_i
		-\tau\ddot x_i)\frac{\partial L}{\partial\dot x_i}
		-\dot A+\sum_{i=1}^n(\xi_i-\tau\dot x_i)\frac{d}{dt}\frac{\partial L}{\partial\dot x_i}.
	\]
Теперь распишем \eqref{eq:dotI}, применив выражение для $\mbfsansE_i(L)$ и
применим~\eqref{eq:prolongation-v}
	\[
	&\sum_{i=1}^n(\xi_i-\tau\dot x_i)\left(\frac{d}{dt}\frac{\partial L}{\partial\dot x_i}
	 	-\frac{\partial L}{\partial x_i}\right)={}\\
	&\qquad
		=\sum_{i=1}^n(\xi_i-\tau\dot x_i)\left(\frac{d}{dt}\frac{\partial L}{\partial\dot x_i}\right)
		+\sum_{i=1}^n\tau\dot x_i\frac{\partial L}{\partial x_i}
		+\tau\frac{\partial L}{\partial t}+\sum_{i=1}^n(\dot \xi_i-\dot\tau\dot x_i)\frac{\partial L}{\partial\dot x_i}
	 	+L\dot\tau-\dot B,
	\]
учитывая, что
	\[
	L=\sum_{i=1}^n(\frac{\partial L}{\partial\dot x_i})\dot x_i+\frac{\partial L}{\partial t},
	\]
а также
	\[
	\sum_{i=1}^n\ddot x_i\frac{d}{dt}\frac{\partial L}{\partial\dot x_i}
		+\sum_{i=1}^n\dot x_i\frac{d}{dt}\frac{\partial L}{\partial {x_i}}
		=\frac{d}{dt}\left(\sum_{i=1}^n\frac{\partial L}{\partial\dot x_i}\dot x_i\right),
	\]
приходим к~утверждению теоремы.


\end{proof}

Теорема Нётер устанавливает соответствие между обобщёнными симметриями
вариационных задач и~законами сохранения (первыми интегралами) уравнений
Эйлера~— Лагранжа. Из теоремы непосредственно вытекает, что $\dot
I=0\bmod\mbfsansE(L)=0$, и, следовательно, $I$ является первым интегралом
уравнений Эйлера~— Лагранжа.

\begin{thebibliography}{0}
\bibitem{bib:1} \url{http://mech.math.msu.su/~shvetz/coronavirus/Noether.pdf}.
\bibitem{bib:2} \url{http://mech.math.msu.su/~shvetz/coronavirus/Prolongation.pdf}.
\bibitem{bib:3} \url{http://alexandr4784.narod.ru/olwer/olw_gl5_3.pdf}.
\end{thebibliography}

\end{document}
