\documentclass[a4paper,11pt]{article}
\usepackage{luatex85}
\usepackage[babelshorthands]{polyglossia}
\usepackage[fleqn,reqno]{amsmath}
\usepackage{amsthm}
\RequirePackage[
	backend=biber,
	bibstyle=gost-numeric,
	citestyle=gost-numeric,
	%defernumbers=true,
	defernumbers=false,
	language=auto,
	autolang=langname,
	]{biblatex}
\usepackage[math-style=ISO,bold-style=upright,partial=italic]{unicode-math}
\usepackage{microtype}
\usepackage[width=16cm,height=24cm]{geometry}
\usepackage[russian]{hyperref}
\usepackage{luamplib}

\setmainlanguage{russian}

\setmainfont{Cambria}
\setsansfont{Calibri}
\setmonofont{Source Code Pro}[Scale=MatchLowercase]
\setmonofont{Source Code Pro}[Scale=MatchLowercase]
\setmathfont{Cambria Math}[sans-style=literal]
\setmathfont{XITS Math}[range={\mathscr,\leqslant,\geqslant}]

\addbibresource{\jobname.bib}

\makeatletter

\allowdisplaybreaks[4]

\def\[#1\]{\begin{align*}#1\end{align*}}
\newcommand\eqtag[1]{\refstepcounter{equation}\tag{\theequation}\label{#1}}

%\newcommand\slashfrac[2]{{#1\fracslash#2}}
\newcommand\slashfrac[2]{{#1/#2}}

\newcommand\pr{\operatorname{\textbf{\textup{pr}}}}

\theoremstyle{definition}
\newtheorem{theorem}{Теорема}
\newtheorem{lemma}{Лемма}
\newtheorem{definition}{Определение}
\newtheorem{example}{Пример}
%\newtheorem{proof}{Доказательство}

\newcommand\metasetup{\hypersetup{
	pdftitle=\@title,
	pdfauthor=\@author,
	linkbordercolor={0 .5 .25},
	}}

\setlength\overfullrule{5pt}

\makeatother

\begin{document}

\hyphenation{Ла-гран-жа ла-гран-же-вой}

\title{Малые колебания консервативных лагранжевых систем}
\author{А.~Н.~Швец}

\metasetup
\maketitle

\section{Состояния равновесия}

Рассмотрим консервативную лагранжеву систему, заданную лагранжианом $L(q,\dot
q)=T(q,\dot q)-U(q)$ (не зависящим от времени). \emph{Положением равновесия\/}
такой системы назовём решение уравнений Эйлера~— Лагранжа вида
$q(t)=q^*=\mathup{const}$, \emph{состоянием равновесия\/}~— пару $(q,\dot
q)=(q^*,0)$. В~дальнейшем без ограничения общности считаем, что эта константа
нулевая, то есть положение равновесия находится в начале координат; если это не
так, то сдвинем соответствующим образом координаты.

\section{Линеаризация лагранжевых уравнений}

Во многом поведение решений лагранжевых уравнений вблизи состояния равновесия
сходно с поведением решений системы линеаризованных уравнений. Для выполнения
линеаризации следует представить уравнения Лагранжа в~виде системы уравнений
первого порядка (ввести переменные $v=\dot q$), а~затем разложить правые части
полученных уравнений в~тейлоровские ряды в~окрестности состояния равновесия
$(0,0)$, отбросив члены разложения степени выше первой.

Однако этот универсальный подход, годящийся не только для лагранжевых
уравнений, но и~для вообще любых, можно значительно упростить, полагаясь на
вариационную специфику уравнений. Заметим, во"=первых, что при линеаризации
лагранжевых уравнений снова получаются лагранжевы, но только линейные
(проверьте!). Во"=вторых, лагранжианам, квадратичным по скоростям отвечают
линейные уравнения, и~наоборот, линейные лагранжевы уравнения отвечают
квадратичным лагранжианам (тоже проверьте!). Всё это позволяет свести операцию
линеаризации лагранжевых уравнений к манипуляциям лишь с~одной функцией
Лагранжа (вместо манипуляций со~многими правыми частями уравнений).

Нужно разложить в~тейлоровский ряд сам лагранжиан $L$ в окрестности состояния
равновесия по отношению ко~всем фазовым переменным $(q,\dot q)$, отбрасывая
члены разложения степени выше \emph{второй.} По полученному в~результате
лагранжиану $\tilde L$ уже нужно строить уравнения, они заведомо будут
линейными.

В задачах механики, когда
$L=\slashfrac12\cdot\sum_{i,j}a_{ij}(q)\dot q_i\dot q_j-U(q)$, получаем
	\[
	\tilde L=\frac12\sum_{i,j}a_{ij}(0)\dot q_i\dot q_j
		-\left(U(0)+\sum_k\frac{\partial U(0)}{\partial q_k}q_k
		+\frac12\sum_{l,m}\frac{\partial^2U(0)}{\partial q_l\partial q_m}q_lq_m\right).
	\]
С учётом того, что в положении равновесия $\slashfrac{\partial U(0)}{\partial
q_k}=0$, а константа $U(0)$ является калибровочной, получим
	\[
	\tilde L=\frac12\sum_{i,j}a_{ij}(0)\dot q_i\dot q_j
		-\frac12\sum_{i,j}\frac{\partial^2U(0)}{\partial q_i\partial q_j}q_iq_j.
	\]
Введём обозначения: $A_{ij}=a_{ij}(0)$~— матрица кинетической энергии,
$B_{ij}=\slashfrac{\partial^2U(0)}{\partial q_i\partial q_j}$~— матрица (Гессе)
потенциальной энергии. Тогда (проверьте!)
	\[
	-\mbfsansE_i(\tilde L)=\sum_sA_{is}\ddot q_s+\sum_sB_{is}q_s.
	\]

Наша следующая забота~— линейными заменами координат придать полученным
уравнениям $\mbfsansE(\tilde L)=0$ наиболее простой вид. Заметим, что обе
матрицы, $A$ и $B$, симметрические, и~к~тому~же матрица~$A$
определённо"=положительная (последнее вытекает из определённой положительности
кинетической энергии). Линейные преобразования координат $q$ продолжаются до
линейных (и~тех~же самых) преобразований скоростей $\dot q$. Как известно из
линейной алгебры, линейными преобразованиями можно одновременно привести пару
квадратичных форм, одна из которых определённо"=положительна, к диагональному
виду, причём определённо"=положительная превратится в сумму квадратов. Для
доказательства нужно задать в линейном пространстве евклидову структуру
с~помощью определённо"=положительной формы, тогда утверждение сводится к
теореме о приведении квадратичной формы к главным осям. Полученные оси можно
выбрать к тому~же ортогональными относительно скалярного произведения~— факт,
который мы вспоминали в~связи с~главными осями инерции.

В новых переменных $\hat q_i=\sum_k C_{ik}q_k$ линеаризованные уравнения
Лагранжа примут вид:
	\[
	\ddot{\hat q_i}+\omega_i^2\hat q_i=0,
	\quad
	i=1,\ldots,n,
	\]
где $\omega_i^2$ суть собственные числа пары форм с~матрицами $A$ и~$B$.
Диагонализующие координаты называются \emph{нормальными.} Нормализация привела
к полному расцеплению линеаризованных уравнений~— каждое уравнение содержит
лишь свою координату.

Если все $\omega_i^2>0$ (это так, если потенциальная энергия имеет
невырожденный минимум в~положении равновесия), то каждая нормальная координата
в~силу линеаризованных уравнений будет совершать гармонические колебания
с~частотой $\omega_i$. В~этом случае движения линеаризованной системы называют
\emph{малыми колебаниями.} Сами линеаризованные уравнения называются
\emph{уравнениями малых колебаний}. Это совсем не означает малость амплитуд
колебаний~— для линейных уравнений она может быть как угодно велика. Дело
в~том, что для исходных уравнений $\mbfsansE(L)=0$ вдали от состояния
равновесия линеаризация $\mbfsansE(\tilde E)=0$ перестаёт быть хорошим
приближением~— отсюда слово \emph{малые.}

Траектории малых колебаний суть так называемые \emph{фигуры Лиссажу́.} Они
в~типичном случае незамкнуты. Однако, если частоты соизмеримы, то~есть линейно
зависимы над~$\BbbZ$, они замкнуты, а~малые колебания периодичны~— период
равен наименьшему общему кратному периодов колебаний отдельных координат.

\section{Малые колебания при наложении голономных связей}

Пусть на лагранжеву систему, способную совершать малые колебания в положении
равновесия, наложена ещё одна дополнительная голономная автономная связь
с~уравнением $f(q)=0$, причём имеющееся положение равновесия $q^*$
удовлетворяет уравнению связи. Связь съедает одну степень свободы. Что можно
сказать о~малых колебаниях с учётом дополнительной связи?

Теперь следует ограничить уравнения малых колебаний на гиперплоскость, заданную
линеаризованным уравнением связи.

Оказывается, полученная система на гиперплоскости будет также совершать
гармонические колебания с частотами $\lambda_1,\ldots,\lambda_{n-1}$, причём
частоты $\omega_1,\ldots,\omega_n$ колебаний исходной системы будут
чередоваться с частотами системы на гиперплоскости:
	\[
	\omega_1\leqslant\lambda_1\leqslant\omega_2\leqslant\lambda_2
		\leqslant\ldots\leqslant\lambda_{n-1}\leqslant\omega_n.
	\]
(здесь предполагается, что все частоты упорядочены по неубыванию). Это следует
из теоремы Куранта (линейная алгебра). Геометрическая интерпретация этой
теоремы: при центральном сечении эллипсоида с~полуосями
$a_1\leqslant\ldots\leqslant a_n$ получается эллипсоид коразмерности один
с~полуосями $b_1\leqslant\ldots\leqslant b_{n-1}$, причём
	\[
	a_1\leqslant b_1\leqslant a_2\leqslant\ldots\leqslant b_{n-1}\leqslant a_n.
	\]
В~частности, для двумерного эллипсоида его центральное круглое сечение обязано
проходить через среднюю ось.

В~технической механике данный результат очень важен. Добавочные связи в~технике
реализуются как дополнительные рёбра жёсткости, балки, фермы, и~т.~п. Их
добавление в~конструкцию, таким образом, может разве что сузить диапазон
собственных частот. Это хорошо, поскольку системы отзывчивы к внешним
возмущениям, имеющим частоты, кратные или близкие к кратным собственных
частот~— это явление \emph{резонанса,} при котором рота солдат, шагающая по
мосту в~ногу (что категорически запрещено), может разрушить мост, если частота
вынуждающего воздействия окажется близкой к~какой"=то из собственных. Чем у́же
диапазон собственных частот, тем лучше, тем жёстче конструкция.

\emph{Челлендж:} предлагаю желающим отличиться раскопать теорему Куранта,
изложить её с~доказательством понятным языком, и~мы выложим на сайт.

\section{Задачи}

Задачи относятся к~предыдущей теме. В~следующий раз, если доживём
({\fontspec{Free Serif}☠}), будут задачи на малые колебания. К~этому времени
нужно изучить также предисловие к~разделу~3 задачника.

\begin{enumerate}

\item
Найдите геодезические на плоскости Лобачевского в~модели на верхней
полуплоскости как графики функций $y(x)$. Обратите внимание, что геодезические,
не представимые в~указанном виде, будут потеряны. Теперь поменяйте местами
в~функционале длины зависимую и независимую переменные и снова найдите
геодезические, в~том числе потерянные. Используйте метод Рауса (где возможно)
и~не забывайте про интеграл Якоби. Фукнционал длины в~обоих случаях:
	\[
	\mscrL_1[y]=\int\frac{\sqrt{1+y'^2}}y\,dx,
	\quad
	\mscrL_2[x]=\int\frac{x'\sqrt{1+\slashfrac1{x'}^2}}y\,dy,
	\]
(штрих означает производную по независимой переменной). Убедитесь сначала, что
$\mscrL_2$ совпадает с~$\mscrL_1$, но только выражен в~других переменных.

\item
2.20.

\item
2.33.

\item
2.35.

\item
2.36.

\item
\emph{Челлендж!}

\end{enumerate}

Теперь задачи на малые колебания. Выполняя линеаризацию (отбрасывая члены
степени выше второй в~лагранжиане), строго обосновывайте, что степень малости
отбрасываевых членов достаточно велика. Теоретическая основа в~этих задачах
ничтожна. Главное~— аккуратность в~вычислениях и~обоснования. Проверяйте
совпадение с~ответом самостоятельно. Решения без обоснований проверять не буду.
Не возражаю против использования систем компьютерной алгебры (Maxima, Reduce,
Wolfram Mathematica), и~даже приветствую.

\begin{enumerate}

\item
3.1, 3.2 (в~сущности, это одна и та~же задача). Исследуйте все возможные
положения равновесия и их характер при разнообразных комбинациях значений
параметров.

\item
3.6.

\item
3.12.

\item
3.14.

\item
3.23.

\end{enumerate}

Будьте здоровы!

\end{document}
