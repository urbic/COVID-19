\documentclass[a4paper,11pt]{article}
\usepackage{luatex85}
\usepackage[babelshorthands]{polyglossia}
\usepackage[fleqn,reqno]{amsmath}
\usepackage{amsthm}
\RequirePackage[
	backend=biber,
	bibstyle=gost-numeric,
	citestyle=gost-numeric,
	%defernumbers=true,
	defernumbers=false,
	language=auto,
	autolang=langname,
	]{biblatex}
\usepackage[math-style=ISO,bold-style=upright,partial=italic]{unicode-math}
\usepackage{microtype}
\usepackage[width=16cm,height=24cm]{geometry}
\usepackage[russian]{hyperref}
\usepackage{luamplib}

\setmainlanguage{russian}

\setmainfont{Cambria}
\setsansfont{Calibri}
\setmonofont{Source Code Pro}[Scale=MatchLowercase]
\setmonofont{Source Code Pro}[Scale=MatchLowercase]
\setmathfont{Cambria Math}[sans-style=literal]
\setmathfont{XITS Math}[range={\mathscr}]

\addbibresource{\jobname.bib}

\makeatletter

\allowdisplaybreaks[4]

\def\[#1\]{\begin{align*}#1\end{align*}}
\newcommand\eqtag[1]{\refstepcounter{equation}\tag{\theequation}\label{#1}}

%\newcommand\slashfrac[2]{{#1\fracslash#2}}
\newcommand\slashfrac[2]{{#1/#2}}

\newcommand\pr{\operatorname{\textbf{\textup{pr}}}}

\theoremstyle{definition}
\newtheorem{theorem}{Теорема}
\newtheorem{lemma}{Лемма}
\newtheorem{definition}{Определение}
\newtheorem{example}{Пример}
%\newtheorem{proof}{Доказательство}

\newcommand\metasetup{\hypersetup{
	pdftitle=\@title,
	pdfauthor=\@author,
	linkbordercolor={0 .5 .25},
	}}

\setlength\overfullrule{5pt}

\makeatother

\begin{document}

\hyphenation{Ла-гран-жа ла-гран-же-вой}

\title{Метод Рауса}
\author{А.~Н.~Швец}

\metasetup
\maketitle

\section{Первые интегралы уравнений Эйлера~— Лагранжа}

Вариационные уравнения Лагранжа $\mbfsansE(L)=0$ при наличии определённых
симметрий лагранжиана обладают первыми интегралами, то~есть такими функциями
$I(t,q,\dot q)$, что $\dot I=0\bmod\mbfsansE(L)=0$. Здесь $\mbfsansE$~—
оператор Эйлера, отображающий лагранжиан~$L$ в~набор уравнений Эйлера~—
Лагранжа.

В~частности, если лагранжиан $L(t,q,\dot q)$ инвариантен относительно сдвигов
какой"=нибудь обобщённой координаты, то~есть $\slashfrac{\partial L}{\partial
q_i}=0$, то функция $p_i=\slashfrac{\partial L}{\partial\dot q_i}$ является
первым интегралом. Это напрямую следует из $i$-го уравнения Эйлера~— Лагранжа.
Обобщённая координата $q_i$ называется \emph{циклической,} а~отвечающий ей
первый интеграл называется \emph{циклическим импульсом.}

\section{Связь уравнений Эйлера~— Лагранжа и~уравнений Гамильтона}

Уравнения Эйлера~— Лагранжа можно записать в~эквивалентной форме~— в~форме
канонических уравнений Гамильтона. Нужно составить функцию Гамильтона
(гамильтониан), выполнив \emph{преобразование Лежандра\/} лагранжиана как
функции обобщённых скоростей. Для этого составляем алгебраические уравнения
	\[
	p_i=\frac{\partial L}{\partial\dot q_i}
	\]
и~разрешаем их относительно скоростей:
	\[
	\dot q_i=f_i(t,q,p)
	\]
(предполагая, что уравнения разрешимы, хотя~бы локально). Затем в~выражение
	\[
	\sum_{i=1}^n\dot q_ip_i-L(t,q,\dot q)
	\]
подставляем полученные выражения скоростей, и, таким образом, получаем функцию
Гамильтона~$H(t,q,p)$. Уравнения
	\[
	\dot p=-\frac{\partial H}{\partial q},
	\quad
	\dot q=\frac{\partial H}{\partial p}
	\]
и есть канонические уравнения Гамильтона. Это $2n$ уравнений первого порядка.

\section{Метод Рауса}

Известный непостоянный первый интеграл общей системы обыкновенных
дифференциальных уравнений
	\[
	\dot x=X(t,x)
	\]
позволяет понизить порядок системы на единицу. Благодаря особой вариационной
природе уравнений Эйлера~— Лагранжа (и эквивалентных им уравнений Гамильтона)
первый интеграл имеет двойную ценность~— он позволяет понизить порядок на две
единицы, исключив тем самым одну степень свободы.

Более сложным является вопрос о возможности понижения порядка (редукции) сразу
на $2k$ единиц при наличии $k$ независимых первых интегралов для уравнений
Эйлера~— Лагранжа. Согласно теореме Нётер каждый первый интеграл лагранжевых
уравнений обусловлен какой"=то одномерной группой Ли симметрий (\emph{группами
Ли\/} называются гладкие многообразия с групповой структурой, для которой
групповая операция задаётся гладкими функциями). Наличие нескольких независимых
первых интегралов означает наличие многомерной группы симметрий. В~случае
абелевой группы симметрий описанная редукция возможна; в~неабелевом случае уже
не обязательно.

Имеются различные методы понижения порядка уравнений Эйлера~— Лагранжа в случае
наличия первых интегралов. Если первые интегралы являются циклическими, очень
удобен \emph{метод Рауса.} При $k$ циклических первых интегралах
соответствующая группа симметрий есть группа трансляций циклических переменных,
она абелева, так что метод Рауса позволяет понизить порядок сразу на
$2k$~единиц.

Будем считать, что первые~$k$ обобщённых координат циклические. Остальные
назовём \emph{позиционными,} так что $q=(q_{\text{cyc}},q_{\text{pos}})$.
Выполним частичное преобразование Лежандра лагранжиана~— по отношению не
ко~всем скоростям (как при составлении гамильтониана), а~лишь по отношению
к~циклическим: в~выражение
	\[
	\sum_{\text{cyc}}\dot q_{\text{cyc}}p_{\text{cyc}}
		-L(t,q_{\text{pos}},\dot q_{\text{cyc}},\dot q_{\text{pos}})
	\eqtag{eq:routh-func}
	\]
подставим вместо циклических скоростей $\dot q_{\text{cyc}}$ их выражения через
циклические импульсы, полученные при решении уравнений
	\[
	p_{\text{cyc}}=\frac{\partial L}{\partial\dot q_{\text{cyc}}},
	\eqtag{eq:cyc-solve}
	\]
относительно циклических скоростей:
	\[
	\dot q_{\text{cyc}}=f_i(t,q_{\text{pos}},\dot q_{\text{pos}},p_{\text{cyc}})
	\]
(правые части не содержат $q_{\text{cyc}}$, так~как они отсутствуют
в~\eqref{eq:cyc-solve}). Построенная в~результате такой подстановки
в~\eqref{eq:routh-func} функция $R(t,q_{\text{pos}},\dot
q_{\text{pos}},p_{\text{cyc}})$ называется \emph{функцией Рауса.}

Функция Рауса сочетает в~себе свойства как функции Лагранжа (по отношению
к~позиционным переменным), так и~функции Гамильтона (по отношению
к~циклическим). Это позволяет переписать исходные уравнения Эйлера~— Лагранжа
в~эквивалентной гибридной форме: по отношению к~позиционным переменным это
будут уравнения Эйлера~— Лагранжа \eqref{eq:routh-eq-lagrange}, а~по отношению
к~циклическим~— уравнения Гамильтона \eqref{eq:routh-eq-hamilton}, где функция
Рауса будет играть роль функции и~Лагранжа и~Гамильтона одновременно:
	\[
	&\frac{\partial R}{\partial q_{\text{pos}}}
		-\left(\frac{\partial R}{\partial\dot q_{\text{pos}}}\right)
		\dot{\vphantom{\Big|}}=0,
	\eqtag{eq:routh-eq-lagrange}\\
	&\dot p_{\text{cyc}}=-\frac{\partial R}{\partial q_{\text{cyc}}}=0,
	\quad
	\dot q_{\text{cyc}}=\frac{\partial R}{\partial p_{\text{cyc}}}.
	\eqtag{eq:routh-eq-hamilton}
	\]
Первый набор уравнений \eqref{eq:routh-eq-hamilton}, как и ожидалось, выражает
постоянство циклических импульсов. Уравнения \eqref{eq:routh-eq-lagrange}
представляют собой уравнения Эйлера~— Лагранжа с~$n-k$ степенями свободы,
содержащие набор параметров $p_{\text{cyc}}$. Решив их, получим зависимости
$q_{\text{pos}}(t)$. Подставив эти зависимости во второй набор уравнений
\eqref{eq:routh-eq-hamilton}, получим уравнения
	\[
	\dot q_{\text{cyc}}=\frac{\partial R(t,q_{\text{pos}}(t),
		\dot q_{\text{pos}}(t),p_{\text{cyc}})}{\partial p_{\text{cyc}}},
	\]
которые немедленно решаются:
	\[
	q_{\text{cyc}}(t)=\int\frac{\partial R(t,q_{\text{pos}}(t),
		\dot q_{\text{pos}}(t),p_{\text{cyc}})}{\partial p_{\text{cyc}}}
		\,dt.
	\]
Таким образом, решение уравнений Эйлера~— Лагранжа с $n$ степенями свободы
и~с~$k$ циклическими координатами сводится к~решению уравнений Эйлера~—
Лагранжа с~$n-k$ степенями свободы.

В~частности, уравнения Эйлера~— Лагранжа с~двумя степенями свободы, одной
циклической координатой и~с~лагранжианом, не зависящим от времени, могут быть
полностью проинтегрированы. Дело в том, что в этом случае функция Рауса также
не зависит от времени, поэтому лагранжева группа уравнений Рауса представляет
собой одно автономное уравнение Эйлера~— Лагранжа относительно единственной
позиционной координаты:
	\[
	\frac{\partial R}{\partial q_{\text{pos}}}
		-\left(\frac{\partial R}{\partial\dot q_{\text{pos}}}\right)\dot{\vphantom{\Big|}}=0,
	\]
обладающее интегралом энергии
	\[
	\dot q_{\text{pos}}\frac{\partial R}{\partial\dot q_{\text{pos}}}-R.
	\]
Условие постоянства этого первого интеграла есть автономное дифференциальное
уравнение первого порядка, в котором можно разделить переменные.

Обратите внимание, что наше определение функции Рауса отличается знаком от
определения в~задачнике. Это не имеет никакого значения для уравнений Рауса.
Мы~же хотели таким выбором знака подчеркнуть связь построения функции Рауса
с~преобразованием Лежандра в~его общепринятой форме.

\section{Преобразование Лежандра полиномов}

В~задачах теоретической механики функции Лагранжа всегда являются квадратными
полиномами относительно скоростей, поскольку зависимость от скоростей в~них
приходит из кинетической энергии. Учитывая это, можно упростить работу при
вычислении преобразования Лежандра лагранжиана при составлении функции Рауса
или Гамильтона.

Преобразование Лежандра функции $f(x)$ есть функция $g(y)$, вычисленная как
	\[
	\sum_{i=1}^nx_iy_i-f(x),
	\]
где все переменные $x$ заменяются на их выражения, полученные при решении
уравнений $y=\slashfrac{\partial f}{\partial x}$ относительно $x$ (опять~же
предполагаем разрешимость этих уравнений).

Функция $\varphi(x)$ называется \emph{однородной\/} степени $r$, если для
любого положительного числа $\lambda$ выполняется $f(\lambda
x)\equiv\lambda^rf(x)$.

Теорема Эйлера об однородных функциях утверждает, что
	\[
	\sum_{i=1}^nx_i\frac{\partial f}{\partial x_i}\equiv rf,
	\]
если $f$~— однородная функция степени~$r$.

Рассмотрим функцию Лагранжа как функцию от тех скоростей, по отношению к
которым выполняется преобразование Лежандра (зависимость от прочих переменных
не принимаем во~внимание). Обозначим как $L_2$, $L_1$ и $L_0$ соответственно
однородные квадратичную, линейную части функции Лагранжа и часть, не содержащую
указанных скоростей. Тогда $L=L_0+L_1+L_2$ и
	\[
	\sum_{i=1}^n\dot q_i\frac{\partial(L_0+L_1+L_2)}{\partial\dot q_i}
		-(L_0+L_1+L_2)
		=0L_0+1L_1+2L_2-L_0-L_1-L_2=L_2-L_0.
	\]
Таким образом при вычислении преобразования Лежандра квадратичных по скоростям
лагранжианов нужно оставить квадратичную часть, выкинуть линейную часть
и~поменять знак у~части, не содержащей скоростей. После этого в~полученном
выражении следует подставить вместо скоростей их выражения через время,
обобщённые координаты и импульсы, заданные неявно уравнениями
$p=\slashfrac{\partial L}{\partial\dot q}$.

\section{Задачи}

\begin{enumerate}

\item
Докажите теорему Эйлера об~однородных функциях (воспользуйтесь определением
однородной функции и~продифференцируйте его, положив $\lambda=1$).

\item
Геодезические на плоскости со~стандартной метрикой можно определить как
стационарные точки лагранжева функционала длины $\int\sqrt{1+y'^2}\,dx$. Найдите
решение соответствующих уравнений Эйлера~— Лагранжа методом Рауса.

Запараметризованные временем те~же геодезические суть стационарные точки
функционала длины $\int\sqrt{\dot x^2+\dot y^2}\,dt$. Попытайтесь решить
соответствующую лагранжеву задачу также методом Рауса. Не выходит? Объясните,
в~чём дело.

\item
Покажите, что уравнения Эйлера~— Лагранжа, описывающие движение материальной
точки по инерции по риманову многообразию в отсутствие трения, совпадают
с~уравнениями геодезических для этого многообразия (известные из курса
дифференциальной геометрии). В этой задаче лагранжиан есть кинетическая энергия
$L=\slashfrac12\cdot g_{ij}(q)\dot q^i\dot q^j$.

\item
Точка движется по прямому круговому конусу с~вертикальной осью и~углом между
осью и~образующей $45^\circ$ без трения под действием силы тяжести ($g=1$).
Решите уравнения движения методом Рауса, взяв в~качестве обобщённых полярные
координаты проекции точки на горизонтальную плоскость.

\item
Решите предыдущую задачу без силы тяжести (движение по инерции), тем самым
будут найдены геодезические для прямого кругового конуса. Конус~—
развёртывающаяся поверхность, его развёртывание~— изометрия на плоскость,
поэтому при развёртывании геодезические должны перейти в~прямые. Проверьте это,
развернув конус в~плоский угол, и~получив полярное представление прямых на
плоскости.

\end{enumerate}

\end{document}
