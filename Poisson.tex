\documentclass[a4paper,11pt]{article}
\usepackage{luatex85}
\usepackage[babelshorthands]{polyglossia}
\usepackage[fleqn,reqno]{amsmath}
\usepackage{amsthm}
\RequirePackage[
	backend=biber,
	bibstyle=gost-numeric,
	citestyle=gost-numeric,
	%defernumbers=true,
	defernumbers=false,
	language=auto,
	autolang=langname,
	]{biblatex}
\usepackage[math-style=ISO,bold-style=upright,partial=italic]{unicode-math}
\usepackage{microtype}
\usepackage[width=16cm,height=24cm]{geometry}
\usepackage[russian]{hyperref}
\usepackage{luamplib}

\setmainlanguage{russian}

\setmainfont{Cambria}
\setsansfont{Calibri}
\setmonofont{Source Code Pro}[Scale=MatchLowercase]
\setmonofont{Source Code Pro}[Scale=MatchLowercase]
\setmathfont{Cambria Math}[sans-style=literal]
\setmathfont{XITS Math}[range={\mathscr}]

\addbibresource{\jobname.bib}

\makeatletter

\allowdisplaybreaks[4]

\def\[#1\]{\begin{align*}#1\end{align*}}
\newcommand\eqtag[1]{\refstepcounter{equation}\tag{\theequation}\label{#1}}

%\newcommand\slashfrac[2]{{#1\fracslash#2}}
\newcommand\slashfrac[2]{{#1/#2}}

\newcommand\pr{\operatorname{\textbf{\textup{pr}}}}

\theoremstyle{definition}
\newtheorem{theorem}{Теорема}
\newtheorem*{theorem*}{Теорема}
\newtheorem{lemma}{Лемма}
\newtheorem{definition}{Определение}
\newtheorem{example}{Пример}
%\newtheorem{proof}{Доказательство}

\newcommand\metasetup{\hypersetup{
	pdftitle=\@title,
	pdfauthor=\@author,
	linkbordercolor={0 .5 .25},
	}}

\setlength\overfullrule{5pt}

\makeatother

\begin{document}

\hyphenation{Ла-гран-жа ла-гран-же-вой}

\title{Скобки Пуассона}
\author{А.~Н.~Швец}

\metasetup
\maketitle

\section{Алгебры Ли}

\emph{Алгеброй Ли\/} называют векторное пространство $\mfrakg$, снабжённое
билинейным отображением $[\cdot,\cdot]\colon\mfrakg^2\to\mfrakg$ таким, что для
любых векторов $\mbfu,\mbfv,\mbfw\in\mfrakg$ выполняется
\begin{enumerate}
\item
$[\mbfv,\mbfw]=-[\mbfw,\mbfv]$.

\item
$[\mbfu,[\mbfv,\mbfw]]+[\mbfv,[\mbfw,\mbfu]]+[\mbfw,[\mbfu,\mbfv]]$.
\end{enumerate}

Операция $[\cdot,\cdot]$ называется \emph{скобкой Ли.} Говорят, что элементы
$\mbfv,\mbfw$, для которых $[\mbfv,\mbfw]=0$, \emph{коммутируют.}

Смысл первого условия ясен~— скобка Ли кососимметрична. Второе условие
называется \emph{тождеством Якоби,} его смысл мы постараемся прояснить.

Оператор~$\mscrD$, заданный в~линейном пространстве, снабжённом какой"=то
билинейной операцией $\lBrack\cdot,\cdot\rBrack$ (\emph{алгебре}), называют
\emph{дифференцированием,} если он отвечает правилу Лейбница:
	\[
	\mscrD\lBrack\mbfv,\mbfw\rBrack
		=\lBrack\mscrD\mbfv,\mbfw\rBrack+\lBrack\mbfv,\mscrD\mbfw\rBrack.
	\]
Если зафиксировать вектор $\mbfv$, скобку можно рассматривать как линейный
оператор $[\mbfv,\cdot]\colon\mfrakg\to\mfrakg$. Тождество Якоби вместе
с~условием кососимметричности означает, что указанный оператор является
дифференцированием алгебры Ли (проверьте!). Иными словами, мы определили
действие алгебры Ли на себе дифференцированиями. Линейность и~правило Лейбница
являются ключевыми свойствами дифференцирования, как~бы мы его ни понимали,
и~позволяют перенести дифференциальное исчисление на произвольные кольца
и~алгебры без разностных отношений и~предельных переходов.

Все сведения о~конечномерной алгебре Ли содержатся в~трёхвалентном
\emph{структурном тензоре\/} $c_{ij}^k\mbfe^i\otimes\mbfe^j\otimes\mbfe_k$
таком, что
	\[
	[\mbfe_i,\mbfe_j]=c_{ij}^k\mbfe_k.
	\]
Здесь $\{\mbfe_\alpha\}$ и $\{\mbfe^\beta\}$~— базисы в~$\mfrakg$ и~дуальном
к~$\mfrakg$ пространстве. Элементы матрицы $c_{ij}^k$ называются
\emph{структурными константами\/} алгебры, и~недаром~— они вполне определяют её
структуру. Условия на скобку Ли накладывают на структурные константы
определённые ограничения:
	\[
	c_{ij}^k=-c_{ji}^k,
	\quad
	c_{ij}^kc_{kl}^m+c_{li}^kc_{kj}^m+c_{jl}^kc_{ki}^m=0.
	\]

Важным примером алгебр Ли служат пространства векторных полей на многообразии,
когда в~качестве скобки Ли берётся их коммутатор
$[\mbfv,\mbfw]=\mbfv\circ\mbfw-\mbfw\circ\mbfv$ как дифференциальных операторов.
На первый взгляд, коммутатор выглядит как дифференциальный оператор второго
порядка, но это только на первый взгляд. Проверьте, что он является оператором
\emph{первого\/} порядка, и, таким образом, отождествляется с~векторным полем.
Проверьте также, что выполняется тождество Якоби (кососимметричность и~так
очевидна).

Алгебры Ли часто ассоциируют с~группами Ли. Делается это следующим образом.
Любая группа действует на себе правыми сдвигами $h\mapsto h\cdot g$ ($\cdot$
обозначает групповую операцию). Для группы Ли~$G$ правое действие продолжается
с~помощью дифференциала до действия на касательное расслоение к~группе,
отображающее векторные поля на~$G$ в~векторные поля. Поля, инвариантные
относительно продолженного правого действия, называются
\emph{правоинвариантными.} Оказывается, что правоинвариантные поля образуют
алгебру Ли относительно коммутатора в~качестве скобки Ли (проверьте!). Эта
алгебра и~называется \emph{алгеброй Ли\/} $\mfrakg$ группы Ли~$G$.

Правоинвариантное поле на~$G$ однозначно задаётся своим значением в~единице
группы, и, наоборот, любой касательный в~единице вектор правыми сдвигами можно
разнести по группе, получив правоинвариантное поле. Это соображение позволяет
перенести структуру алгебры Ли правоинвариантных полей на касательное
пространство к~группе в~её единице.

Таким образом, группа Ли определяет свою алгебру. До некоторой степени возможно
восстановить группу по её алгебре. Оказывается, для данной конечномерной
алгебры Ли $\mfrakg$ найдётся единственная связная односвязная группа Ли~$G$,
чьей алгеброй служит~$\mfrakg$. Эта группа служит односвязным накрытием для
любой другой связной группы Ли с~той~же алгеброй Ли. Мы приводим это
утверждение без доказательства.

В~частности, все одномерные алгебры Ли, очевидно, абелевы ($[\cdot,\cdot]=0$),
и~связные группы Ли, ассоциированные с~ними, изоморфны либо аддитивной группе
$(\BbbR,+)$, либо~$\symup{SO}(2)$. Прямая $(\BbbR,+)$ служит односвязной
накрывающей для окружности $\symup{SO}(2)$.

Локальную группу Ли преобразований гладкого многообразия можно восстановить по
её алгебре Ли с~помощью \emph{экспоненциального отображения\/}
$\mbfv\mapsto g=\exp(\mbfv)$. Обратное к~экспоненциальному отображение,
определённое, вообще говоря, лишь вблизи единицы, отображает достаточно малую
окрестность единицы на касательное пространство к~группе в~единице, то~есть на
алгебру.

\section{Скобки Пуассона}

Автономную систему канонических уравнений Гамильтона ($\slashfrac{\partial
H}{\partial t}=0$)
	\[
	\dot p_i=-\frac{\partial H}{\partial q_i},
	\quad
	\dot q_i=\frac{\partial H}{\partial p_i},
	\quad
	i=1,\ldots,n
	\eqtag{eq:canon-eq}
	\]
можно интерпретировать как векторное поле~$\mbfv_H$ на фазовом многообразии
гамильтоновой системы,
	\[
	\mbfv_H=\sum_{i=1}^n\left(-\frac{\partial H}{\partial q_i}\partial_{p_i}
		+\frac{\partial H}{\partial p_i}\partial_{q_i}\right).
	\]
Такие поля имеют очень специальный вид, поскольку задаются одной функцией $H$
(в~общем случае поля на~$2n$"=мерном многообразии определяются $2n$~функциями).
Поля такого вида называются \emph{гамильтоновыми.} Оказывается, гамильтоновы
поля образуют алгебру Ли относительно коммутатора: для двух функций $F(p,q)$,
$G(p,q)$ найдётся функция $H(p,q)$ такая, что
	\[
	[\mbfv_F,\mbfv_G]=\mbfv_H,
	\]
причём
	\[
	H=\sum_{i=1}^n\left(\frac{\partial F}{\partial p_i}\frac{\partial G}{\partial q_i}
		-\frac{\partial F}{\partial q_i}\frac{\partial G}{\partial p_i}\right).
	\]
Выражение для $H$, обычно обозначаемое как $\{F,G\}$\footnote{В~некоторых
книгах (например, в~задачнике) определение скобки Пуассона отличается знаком.
Это не страшно.}, называется \emph{скобкой Пуассона\/} функций $F$ и~$G$. Оно
является билинейным и~кососимметрическим относительно $F$ и~$G$. Более того,
выполняется (проверьте!) тождество Якоби
	\[
	\{F,\{G,H\}\}+\{G,\{H,F\}\}+\{H,\{F,G\}\}=0.
	\]
Всё это делает линейное над~$\BbbR$ пространство гладких функций на
$2n$"=мерном многообразии алгеброй Ли. Учитывая, что соответствие между
гамильтонианом~$H$ и~гамильтоновым векторным полем $\mbfv_H$ линейное, мы можем
утверждать, что имеется гомоморфизм из алгебры Ли функций Гамильтона в~алгебру
Ли гамильтоновых векторных полей. У~этого гомоморфизма ненулевое ядро
(см.~задачу~\ref{prb:1}).

Как легко проверить,
	\[
	\mbfv_F(G)=-\mbfv_G(F)=\{F,G\}.
	\]
С~учётом этого канонические уравнения~\eqref{eq:canon-eq} можно записать в~виде
	\[
	\dot p_i=\{H,p_i\},
	\quad
	\dot q_i=\{H,q_i\}.
	\]
Вообще, для произвольной функции $F(p,q)$
	\[
	\dot F=\{H,F\}
	\]
в~силу канонических уравнений (проверьте!). В~частности, если $\{H,F\}=0$, то
$F$ является первым интегралом канонических уравнений с~гамильтонианом~$H$,
и~наоборот, $H$ является первым интегралом для канонических уравнений
с~гамильтонианом~$F$. Подчеркнём, что сказанное верно для функций $F$, $H$, не
зависящих от времени. Не составит труда перенести рассуждения на общий случай,
но результат уже будет другим: пусть функции зависят от времени. Тогда
$F(t,p,q)$ служит первым интегралом уравнений с~гамильтонианом~$H$ тогда
и~только тогда, когда
	\[
	\frac{\partial F}{\partial t}+\{H,F\}=0.
	\]

Но вернёмся к~автономному случаю. Легко проверить (проверьте,
задача~\ref{prb:1}В!), что первые интегралы канонических уравнений
с~гамильтонианом~$H$ образуют алгебру Ли относительно скобки Пуассона~—
подалгебру в~алгебре всех гладких функций на фазовом пространстве. Это
позволяет в~редких случаях получать новые первые интегралы как скобки Пуассона
имеющихся.

\section{Задачи}

\begin{enumerate}

\item
Проверьте что алгебра Ли $\symfrak{so}(3)$ группы Ли $\symup{SO}(3)$ изоморфна
алгебре (тоже Ли) $(\BbbR^3,[\cdot,\cdot])$ векторов трёхмерного пространства
с~векторным произведением в~роли скобки Ли. И~алгебре кососимметрических
операторов в~$\BbbR^3$ с~коммутатором в~роли скобки. [Тут впору вспомнить формулу
Пуассона из кинематики твёрдого тела. Немного проясняется, как связана группа
вращений твёрдого тела и~векторное умножение в~формуле Пуассона, правда?]
\emph{Подсказка:\/} если не придумаете чего"=то более изящного, можно найти
структурные тензоры перечисленных алгебр.

\item\label{prb:1}
А.~Докажите, что $[\mbfv_F,\mbfv_G]=\mbfv_{\{F,G\}}$.

Б.~Найдите ядро гомоморфизма из алгебры функций в~алгебру гамильтоновых полей.

В.~Проверьте, что первые интегралы канонических уравнений с~гамильтонианом~$H$,
не зависящие от времени, образуют подалгебру Ли алгебры гладких функций.

\item
Проверьте, что пространство векторных полей на многообразии с~коммутатором в~роли
скобки Ли действительно является алгеброй Ли.

\item
Пусть, как и раньше, $g_\mbfv^\varepsilon=\exp(\varepsilon\mbfv)$~— фазовый поток
поля~$\mbfv$. Ранее мы определяли действие векторных полей на многообразии на
функции
	\[
	(\mbfv(f))(x)=\lim_{\varepsilon\to0}\frac{f(g_\mbfv^\varepsilon(x))-f(x)}\varepsilon
		=\left.\frac d{d\varepsilon}\right|_{\varepsilon=0}
		f(g_\mbfv^\varepsilon(x)).
	\]
как действие дифференциальных операторов~— это просто дифференцирование функции
вдоль векторного поля (не путать с~ковариантной производной!).
Можно попытаться по аналогии определить дифференцирование других тензорных
объектов, например, векторных полей. Вот наивная попытка:
	\[
	\mbfv(\mbfw)\stackrel{?}=\lim_{\varepsilon\to0}
		\frac{\mbfw|_{g_\mbfv^\varepsilon(x)}-\mbfw|_x}\varepsilon.
	\]
Она неудачная, так~как слагаемые в~числителе принадлежат разным
пространствам: касательным к~многообразию в~разных точках~— $x$
и~$g_\mbfv^\varepsilon(x)$, их вычитание не определено. Чтобы определение стало
корректным, нужно перед вычитанием перенести некоторым естественным образом
вектор $\mbfw|_{g_\mbfv^\varepsilon(x)}$ в~точку~$x$. Для этой цели хорошо
подходит дифференциал отображения~$g_\mbfv^{-\varepsilon}$:
	\[
	\mbfv(\mbfw)|_x=\lim_{\varepsilon\to0}
		\frac{(g^{-\varepsilon})^*(\mbfw|_{g_\mbfv^\varepsilon(x)})-\mbfw|_x}\varepsilon
		=\left.\frac d{d\varepsilon}\right|_{\varepsilon=0}
		\left((g^{-\varepsilon})^*(\mbfw|_{g_\mbfv^\varepsilon(x)})-\mbfw|_x\right).
	\]
Покажите, что $\mbfv(\mbfw)=[\mbfv,\mbfw]$.

\item
Покажите, что
	\[
	[\mbfv,\mbfw]|_x
		=\left.\frac d{d\varepsilon}\right|_{\varepsilon=+0}
		\left(g_\mbfw^{-\sqrt\varepsilon}\circ g_\mbfv^{-\sqrt\varepsilon}
		\circ g_\mbfw^{\sqrt\varepsilon}\circ g_\mbfv^{\sqrt\varepsilon}\right)(x).
	\]

\item
\emph{Для тех, кто хочет отличиться.} Покажите, что поля обычных (не
обобщённых) вариационных симметрий лагранжевой задачи $\int L\,dt$ образуют
алгебру Ли. Так можно получать новые вариационные симметрии по имеющимся. [Мы
ограничиваемся обычными симметриями, поскольку не хотим определять коммутатор
обобщённых полей~— как вы, наверное, помните, порождаемые ими преобразования,
вообще говоря, действуют на бесконечном \emph{пространстве струй\/} функций
$x(t)$~— пространстве $\{(x,\dot x,\ddot x,\dddot x,\ldots,x^{(\infty)})\}$.]

\item
5.2.

\item
5.4.

\item
5.5.

\item
5.7.

\item
5.8.

\item
5.22.

\end{enumerate}

\end{document}
