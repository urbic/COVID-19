\documentclass[a4paper,11pt]{article}
\usepackage{luatex85}
\usepackage[babelshorthands]{polyglossia}
\usepackage[fleqn,reqno]{amsmath}
\usepackage{amsthm}
\RequirePackage[
	backend=biber,
	bibstyle=gost-numeric,
	citestyle=gost-numeric,
	%defernumbers=true,
	defernumbers=false,
	language=auto,
	autolang=langname,
	]{biblatex}
\usepackage[math-style=ISO,bold-style=upright,partial=italic]{unicode-math}
\usepackage{microtype}
\usepackage[width=16cm,height=24cm]{geometry}
\usepackage[russian]{hyperref}
\usepackage{luamplib}

\setmainlanguage{russian}

\setmainfont{Cambria}
\setsansfont{Calibri}
\setmonofont{Source Code Pro}[Scale=MatchLowercase]
\setmonofont{Source Code Pro}[Scale=MatchLowercase]
\setmathfont{Cambria Math}[sans-style=literal]
\setmathfont{XITS Math}[range={\mathscr}]

\addbibresource{\jobname.bib}

\makeatletter

\allowdisplaybreaks[4]

\def\[#1\]{\begin{align*}#1\end{align*}}
\newcommand\eqtag[1]{\refstepcounter{equation}\tag{\theequation}\label{#1}}

%\newcommand\slashfrac[2]{{#1\fracslash#2}}
\newcommand\slashfrac[2]{{#1/#2}}

\newcommand\pr{\operatorname{\textbf{\textup{pr}}}}

\theoremstyle{definition}
\newtheorem{theorem}{Теорема}
\newtheorem*{theorem*}{Теорема}
\newtheorem{lemma}{Лемма}
\newtheorem{definition}{Определение}
\newtheorem{example}{Пример}
%\newtheorem{proof}{Доказательство}

\newcommand\metasetup{\hypersetup{
	pdftitle=\@title,
	pdfauthor=\@author,
	linkbordercolor={0 .5 .25},
	}}

\setlength\overfullrule{5pt}

\makeatother

\begin{document}

\hyphenation{Ла-гран-жа ла-гран-же-вой}

\title{Теорема Нётер}
\author{А.~Н.~Швец}

\metasetup
\maketitle

\section{Векторные поля и~их фазовые потоки}

Пусть $M$~— гладкое многообразие с~локальными координатами
$x=(x_1,\ldots,x_n)$. Дифференциальные операторы первого порядка
	\[
	\mbfv=\sum_{i=1}^n\xi_i(x)\partial_{x_i},
	\eqtag{eq:vector-field}
	\]
где $\partial_{x_i}=\slashfrac\partial{\partial x_i}$, отождествляются, как
известно, с~векторными полями на~$M$. \emph{Фазовый поток\/} системы
обыкновенных дифференциальных уравнений
	\[
	\frac{dx_i}{d\varepsilon}=\xi_i(x),
	\quad
	i=1,\ldots,n
	\eqtag{eq:phase-flow}
	\]
есть однопараметрическое семейство отображений $g^\varepsilon$, отображающих
начальное условие $x|_0$ в~решение $x|_\varepsilon$
уравнений~\eqref{eq:phase-flow} в~момент времени~$\varepsilon$. При
определённых предположениях о~функциях $\xi_i(x)$ (в~дальнейшем считаем их
выполненными) фазовый поток существует, по крайней мере, при $\varepsilon$,
достаточно близких к~нулю.

Преобразования $g^\varepsilon$ образуют однопараметрическую абелеву локальную
группу Ли, поскольку, очевидно,
	\[
	g^0=\mathup{id},
	\quad
	g^{\alpha+\beta}=g^\beta\circ g^\alpha,
	\quad
	(g^\alpha)^{-1}=g^{-\alpha},
	\]
по крайней мере, при достаточно малых~$\alpha$, $\beta$. Группа
\emph{локальная,} поскольку групповая структура на множестве
$G=\{g^\varepsilon\}$ определена, вообще говоря, лишь в~окрестности единицы
группы. В~дальнейшем термин «локальный» применительно к~группам будет
опускаться. Убедитесь, что
$g^\varepsilon=\exp(\varepsilon\mbfv)=\sum_{k=0}^\infty\slashfrac{\varepsilon^k\mbfv^k}{k!}$.

Поле $\mbfv$ называется \emph{инфинитезимальной образующей,} или
\emph{генератором\/} группы~$G$.

Действие групповых преобразований $g^\varepsilon$ на~$M$ естественным образом
продолжается до действия на $C^1(M)$~— множество гладких функций на~$M$:
	\[
	(g^\varepsilon(f))(x)=f(g^\varepsilon(x)).
	\]
Тогда, дифференцируя по~$\varepsilon$ и~полагая $\varepsilon=0$, получим
	\[
	\left.\frac d{d\varepsilon}\right|_{\varepsilon=0}f(g^\varepsilon(x))
		=\mbfv(f),
	\eqtag{eq:infinitesimal}
	\]
что и~даёт право отождествить векторные поля с~дифференциальными операторами
первого порядка. Генератор, если можно так выразиться, осуществляет групповое
преобразование при бесконечно малом~$\varepsilon$:
	\[
	g^\varepsilon(f)=f+\varepsilon\mbfv(f)+o(\varepsilon).
	\]

Между прочим, для группы $g^\varepsilon\colon x\mapsto x+\varepsilon$
с~генератором $\partial_x$ получаем формулу Тейлора
	\[
	f(x+\varepsilon)=\exp(\varepsilon\partial_x)(f(x)).
	\]

\section{Вариационные симметрии лагранжевых систем}

Пусть $M=\{x\}$~— конфигурационное многообразие лагранжевой системы,
$M^*=\{(x,\dot x)\}$~— его фазовое многообразие (то~есть касательное расслоение
к~$M$), $L\colon\BbbR\times M^*\to\BbbR$~— лагранжиан, $\mscrL[x]=\int L\,dt$~—
соответствующий лагранжев интегральный функционал.

В~предыдущем разделе мы установили соответствие между векторными полями~$\mbfv$
на~$M$ и~однопараметрическими группами~$g^\varepsilon=\exp(\varepsilon\mbfv)$
преобразований~$M$. Действие $g^\varepsilon$ на координаты~$x$ продолжается
согласованным образом до действия на скорости~$\dot x$:
	\[
	g^\varepsilon(x)=\hat x,
	\quad
	(g^\varepsilon)^*(\dot x)=\hat{\dot x}.
	\]
Здесь $(g^\varepsilon)^*$~— дифференциал преобразования $g^\varepsilon$.
Условия согласованности есть \emph{условия Пфаффа:}
	\[
	(d\hat x_i=\hat{\dot x_i}\,dt)
		\bmod(dx_j=\dot x_j\,dt),
	\quad
	i,j=1,\ldots,n,
	\eqtag{eq:pfaff}
	\]
выражающиеся в~том, что производные по~$t$ при преобразованиях переходят снова
в~производные. Другими словами, требуется, чтобы под действием преобразования
скорость $\dot x$ гладкой кривой на~$M$ в~её точке $x$ переводилась
преобразованием в~скорость $\hat{\dot x}$ преобразованной кривой в~её точке
$\hat x$. Так определяется операция первого \emph{продолжения\/} группы~$G$,
$\operatorname{\symbf{pr}}^{(1)}G$. Аналогичным образом можно построить
и~высшие продолжения группы~$G$, групповые преобразования которых действуют
согласованно на $(x,\dot x,\ddot x,\dddot x,\ldots,x^{(n)})$.

Операция продолжения групповых преобразований $g^\varepsilon$ переносится
в~силу~\eqref{eq:infinitesimal} на генераторы:
	\[
	\operatorname{\symbf{pr}}^{(1)}\mbfv
		=\sum_{i=1}^n\xi_i^{(0)}\partial_{x_i}+\sum_{i=1}^n\xi_i^{(1)}\partial_{\dot x_i},
	\quad
	\xi_i^{(0)}=\xi_i,
	\quad
	\xi_i^{(1)}=\dot\xi_i
	\eqtag{eq:prolongation-v}
	\]
(проверьте, пользуясь условиями Пфаффа~\eqref{eq:pfaff}!).

Заметим, что коэффициенты $\xi$ векторного поля~\eqref{eq:prolongation-v} (они
называются \emph{характеристиками\/}) зависят только от $x$. Ничто не мешает
позволить им зависеть также от~$t$, формулы
продолжения~\eqref{eq:prolongation-v} останутся теми~же.

Дальнейшее обобщение заключается в~том, чтобы позволить характеристикам~$\xi$
зависеть ещё и~от скоростей. Можно показать, что и~в~этом случае формулы
продолжения~\eqref{eq:prolongation-v} останутся в~силе. Однако здесь нужно
заметить, что в~формулах продолжения коэффициенты $\dot\xi$ будут уже зависеть
от $\ddot x$, из-за чего станет невозможным определить действие группы
$\exp(\varepsilon\operatorname{\symbf{pr}}^{(1)}\mbfv)$ ни на $M^*=\{(x,\dot
x)\}$, ни даже на $M^{(k)}=M^{\underbrace{**\cdots*}_k}=\{(x,\dot x,\ddot
x,\ldots,x^{(k)})\}$ для некоторого конечного~$k$. Тем не менее при
определённых оговорках можно определить действие
$\operatorname{\symbf{pr}}^{(1)}G$ на гладких кривых на~$M$.

Конечно, можно обобщать понятие векторного поля и~дальше. Можно позволить
характеристикам зависеть от высших производных. Но такое обобщение не найдёт
полезного применения в~задачах теоретической механики, когда лагранжиан~$L$
зависит самое большее от первых производных. Можно рассмотреть более общие поля
вида
	\[
	\mbfv=\sum_{i=1}^n\xi_i\partial_{x_i}+\tau\partial_t,
	\]
и~получить для них формулы продолжения, обобщающие~\eqref{eq:prolongation-v}
(они оказываются более сложными). Но, как оказывается, такое обобщение ничего
полезного уже не добавит, поскольку поля данного вида будут в~некотором смысле
задавать те~же преобразования, что и~поля с~$\tau=0$. Они в~этом смысле
(который мы не уточняем) будут эквивалентны своим \emph{эволюционным
представителям~—\/} полям вида
	\[
	\tilde\mbfv=\sum_{i=1}^n(\xi_i-\tau\dot x_i)\partial_{x_i}.
	\]

\section{Теорема Нётер}

Итак, рассмотрим обобщённое векторное поле вида~\eqref{eq:vector-field}, где
функции $\xi_i$ зависят от $(t,x,\dot x)$. Назовём его \emph{полем обобщённой
симметрии\/} лагранжевой задачи $\mscrL[x]=\int L(t,x,\dot x)\,dt$, если
найдётся такая дифференциальная функция $A$, что
	\[
	\operatorname{\symbf{pr}}^{(1)}\mbfv(L)=\dot A.
	\eqtag{eq:symmetry-cond}
	\]
Под \emph{дифференциальной функцией\/} мы понимаем функцию независимой
переменной~$t$, зависимых $x_i$ и~производных вплоть до некоторого порядка.

В~оправдание названия полей обобщённых симметрий заметим, что в~своём
инфинитезимальном действии такие поля прибавляют к~лагранжиану калибровочное
слагаемое $\varepsilon\dot A$, что означает добавление к~функционалу~$\mscrL$
постоянного функционала $\varepsilon\int\dot A\,dt$ (добавка зависит лишь от
значений функций $x(t)$ на концах отрезка интегрирования, и, таким образом, не
меняется при варьировании). Это, конечно, никак не сказывается на стационарных
точках $\mscrL$, следовательно, групповые преобразования переводят
решения~$x(t)$ уравнений Эйлера~— Лагранжа в~решения.

\begin{theorem*}[A.~E.~Noether]
Пусть $\mbfv=\sum_{i=1}^n\xi_i\partial_{x_i}$~— поле обобщённой симметрии
лагранжевой задачи~$\mscrL$ и
	\[
	I(t,x,\dot x)=\sum_{i=1}^n\xi_i\frac{\partial L}{\partial\dot x_i}-A.
	\]
Тогда
	\[
	\dot I=-\sum_{i=1}^n\xi_i\mbfsansE_i(L).
	\eqtag{eq:dotI}
	\]
\end{theorem*}

\begin{proof}
Непосредственная проверка с~учётом~\eqref{eq:symmetry-cond}
и~\eqref{eq:prolongation-v} (докажите!).
\end{proof}

Теорема Нётер устанавливает соответствие между обобщёнными симметриями
вариационных задач и~законами сохранения (первыми интегралами) уравнений
Эйлера~— Лагранжа. Из теоремы непосредственно вытекает, что
	\[
	\dot I=0\bmod\mbfsansE(L)=0,
	\]
и, следовательно, $I$ является первым интегралом уравнений Эйлера~— Лагранжа.

Приведённая здесь формулировка теоремы Нётер является более общей, нежели та,
которая обычно приводится в~учебниках по теоретической механике. В~традиционных
(слабых) формулировках ограничиваются обычными (не обобщёнными) полями
симметрий, у~которых характеристики не зависят от скоростей, а~функция~$A$
равна нулю. При таком подходе могут получиться лишь первые интегралы, линейные
по отношению к~скоростям. Другие важные случаи первых интегралов (например,
интеграл Якоби, см.~задачу~\ref{prb:jacobi}) не находят объяснения с~позиций
слабой теоремы Нётер. Сильная теорема связывает любые нетривиальные
интегралы уравнений Эйлера~— Лагранжа с~соответствующими обобщёнными
симметриями лагранжевой задачи. Эта связь между симметриями и~законами
сохранения является ключевой во всех лагранжевых задачах математической физики,
не только в~теоретической механике, но и~в~механике сплошных сред, квантовой
теории поля, геометрии.

\section{Задачи}

\begin{enumerate}

\item
Проверьте, что
\begin{itemize}
\item
При $g^\varepsilon\colon x\mapsto x+\varepsilon$ получается
$\mbfv=\partial_x$~— генератор группы сдвигов~— аддитивной группы $\BbbR$.

\item
При $g^\varepsilon\colon x\mapsto e^\varepsilon x$
получается $\mbfv=\sum_{i=1}^nx_i\partial_{x_i}$~— генератор группы гомотетий~—
мультипликативной группы $\BbbR^+$.

\item
При
$g^\varepsilon\colon(x,y)\mapsto(x\cos\varepsilon-y\sin\varepsilon,x\sin\varepsilon+y\cos\varepsilon)$
получается $\mbfv=-y\partial_x+x\partial_y$~— генератор группы поворотов
$\mathup{SO}(2)$.
\end{itemize}

\item\label{prb:jacobi}
\begin{itemize}
\item
Найдите поле обобщённой симметрии в~случае наличия циклической координаты
($\slashfrac{\partial L}{\partial x_{\text c}}=0$) и~соответствующий в~силу
теоремы Нётер первый интеграл.
\item
Покажите, что обобщённое поле $\mbfv=-\sum_{i=1}^n\dot x_i\partial_{x_i}$
является полем вариационной симметрии лагранжевой задачи в~автономном случае
($\slashfrac{\partial L}{\partial t}=0$). Найдите соответствующий первый
интеграл. Между прочим, указанное поле служит эволюционным представителем поля
$\partial_t$, которое, очевидно, сохраняет лагранжев функционал в~автономном
случае.
\item
В~задаче о~движении точки на плоскости в~центральном поле сил найдите первый
интеграл, отвечающий поворотной симметрии задачи. Объясните механический смысл
интеграла.
\end{itemize}

\item
Пусть $L(x,y,\dot x,\dot y)=\slashfrac{(\dot x^2+\dot y^2)}{\varphi(x,y)}$, где
$\varphi$~— однородная функция степени~$2$ совокупности своих аргументов.
Найдите две обобщённых вариационных симметрии задачи, и, пользуясь
соответствующими двумя нётеровыми первыми интегралами, найдите движение.
\emph{Подсказки:\/}~1)~теорема Эйлера об~однородных функциях! 2)~Исключите $dt$
из условий постоянства нётеровых интегралов, получите уравнение первого порядка
относительно~$y(x)$. Убедитесь, что полученное уравнение однородное, и~решите
его с~помощью уместной в~таких случаях подстановки. А~какая подстановка
уместна? Как известно, однородное уравнение первого порядка $\Phi(x,y,y')=0$
может быть переписано в~равносильной форме через полный набор инвариантов
однородности, в~качестве которых можно взять $u=\slashfrac yx$, $p=y'=u+xu'$,
то~есть как $\tilde\Phi(u,u+xu')=0$. После подстановки переменные разделяются.
3)~Когда найдена траектория, уже нетрудно найти закон движения.

\item
Пусть $L=\slashfrac{(\dot x^2+\dot y^2)}2+\slashfrac k{r^2}$, где
$r^2=x^2+y^2$. Рассмотрите группу преобразований
$(t,x,y)\mapsto(e^{2\varepsilon}t,e^\varepsilon x,e^\varepsilon y)$, убедитесь,
что групповые преобразования сохраняют функционал $\int L\,dt$. Найдите
генератор группы, затем его эволюционный представитель. Проверьте, что
эволюционный представитель является обобщённой вариационной симметрией
лагранжевой задачи (совпадение? не думаю). Найдите соответствующий нётеров
интеграл. Решите задачу, пользуясь нётеровым интегралом совместно с~интегралом
Якоби (траекториями движения должны получиться \emph{спирали Котса,} Cotes
spirals, в~полярных координатах имеющие в~типичном случае вид наподобие
$r=\slashfrac a{{\cos(b\theta+c)}}$).

\item
В~задаче о~движении точки на плоскости по инерции, $T=\slashfrac{(\dot x^2+\dot
y^2)}2$ найдите наиболее общий вид первого интеграла,
пользуясь~\eqref{eq:dotI}. Найдите также общий вид обобщённой симметрии этой
задачи.

\item
Выведите формулу продолжения~\eqref{eq:prolongation-v} 1)~для обычных полей
(характеристики зависят только от~$x$), 2)~для обобщённых полей (характеристики
зависят от $x$, $\dot x$)~— \emph{этот пункт для желающих отличиться}.

\end{enumerate}

\end{document}
