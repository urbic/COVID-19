\documentclass[11pt,a4paper]{article}
\usepackage[babelshorthands]{polyglossia}
\usepackage{amsfonts}
\usepackage{amsmath}
\usepackage{amsthm}
\usepackage{amscd}
\usepackage{amssymb}
\usepackage{mathtools}
\usepackage{microtype}
\usepackage{unicode-math}
\usepackage{indentfirst}
\usepackage[russian]{hyperref}
%\usepackage[left=2cm,right=2cm,
%    top=3cm,bottom=3cm,bindingoffset=0cm]{geometry}
\usepackage[width=16cm,height=24cm]{geometry}

\setdefaultlanguage{russian}
\setmainfont{Cambria}
\setmathfont{Cambria Math}
\setmathfont{XITS Math}[range={\mathscr,\leqslant,\geqslant}]


\begin{document}
\begin{center}
\scshape{\LARGE{Теорема Куранта}}
\vspace{1cm}

\scshape{А. Завадский}
\end{center}
\section{Теорема Куранта}

Введём в пространстве $\mathbb{R}^n$ скалярное произведение: $(x,y)=x^Ty.$

\newtheorem{theorem}{Теорема}
\begin{theorem} [Р. Курант]

Пусть $A$ --- симметрическая вещественная матрица размера $n\times n$, $~\lambda_1\leqslant\lambda_2\leqslant\ldots\leqslant\lambda_n$ --- её собственные значения, расположенные в порядке возрастания. 
Будем обозначать через $L_k$ множество всех $k$"=мерных подпространств в $\mathbb R^{n}$, а его элементы, т.е. подпространства размерности $k,$ --- символом $W$.  Тогда 
$$\lambda_k=\min_{W\in L_{k}}\max_{\substack{x\in W\\\|x\|=1}}(x,Ax)=\max_{W\in L_{n-k+1}}\min_{\substack{x\in W\\\|x\|=1}}(x,Ax).$$
\end{theorem}
\slshape Доказательство.
\upshape

В некотором ортонормированном базисе матрица $A$ диагональна, т.е. $(x,Ax)=\lambda_1x_1^2+\ldots+\lambda_nx_n^2.$ 
Рассмотрим $(n-k+1)$"=мерное подпространство $W_1=\{x~|~x_1=\ldots=x_{k-1}=0\}$ и $k$"=мерное подпространство  $W_2=\{x~|~x_{k+1}=\ldots=x_n=0\}.$
Заметим, что для любого вектора $x\in W_1,$ имеющего единичную норму, 
\begin{gather}
(x,Ax)=\lambda_1x_1^2+\ldots+\lambda_nx_n^2\geqslant\lambda_k(x_k^2+\ldots+x_n^2)=\lambda_k.\label{1}
\end{gather}
Значит, $\lambda_k\leqslant\min\limits_{\substack{x\in W_1\\\|x\|=1}}(x,Ax).$ Тогда $\lambda_k\leqslant\max\limits_{W\in L_{n-k+1}}\min\limits_{\substack{x\in W\\\|x\|=1}}(x,Ax).$

Для любого вектора $x\in W_2, \|x\|=1$ имеем
\begin{gather}
(x,Ax)=\lambda_1x_1^2+\ldots+\lambda_kx_k^2\leqslant\lambda_k(x_1^2+\ldots+x_k^2)=\lambda_k,\label{2}
\end{gather}
т.е. $\lambda_k\geqslant\max\limits_{\substack{x\in W_2\\\|x\|=1}}(x,Ax).$ Тогда $\lambda_k\geqslant\min\limits_{W\in L_{k}}\max\limits_{\substack{x\in W\\\|x\|=1}}(x,Ax).$

Пусть $W_3$ --- произвольное $k$"=мерное подпространство.
Так как $\dim W_1+\dim W_3>n,$ то $W_1\cap W_3\ne\{0\}.$ 
Тогда найдётся вектор $x\in W_1\cap W_3$ такой, что $\|x\|=1$ и 
для него выполнено неравенство \eqref{1}.
Поэтому $\lambda_k\leqslant\max\limits_{x\in W_1\cap W_2}(x,Ax)\leqslant\max\limits_{x\in W_2}(x,Ax).$ 
Подпространство $W_2\in L_k$ мы выбрали произвольным образом, значит $\lambda_k\leqslant\min\limits_{W\in L_{k}}\max\limits_{\substack{x\in W\\\|x\|=1}}(x,Ax).$

Взяв теперь в качестве $W_3$ произвольное $(n-k+1)$"=мерное подпространство и рассмотрев некоторый вектор единичной нормы в $W_2\cap W_3,$ получим аналогично, что $\lambda_k\geqslant\min\limits_{\substack{x\in W_1\\\|x\|=1}}(x,Ax).$ Тогда $\lambda_k\geqslant\max\limits_{W\in L_{n-k+1}}\min\limits_{\substack{x\in W\\\|x\|=1}}(x,Ax).$
\qed

\textbf{Замечание:}
теорема верна и в случае эрмитовой матрицы $A$ и эрмитова скалярного произведения в $\mathbb C^n,$ причём доказательство переносится без изменений.
\section{Малые колебания при наложении голономных связей}

Рассмотрим консервативную лагранжеву систему, заданную лагранжианом $L(q,\dot q)=T(q,\dot q)-U(q),$ способную совершать малые колебания в положении равновесия $q=0$. Пусть $\lambda_1\leqslant\ldots\leqslant\lambda_n$ --- её собственные значения. Наложим на систему дополнительную голономную автономную связь $g(q)=0,$ такую, что $g(0)=0,$ т.е. положение равновесия удовлетворяет уравнению связи. Считаем также, что $g$ --- гладкая функция и $\frac{\partial g}{\partial q}\big{|}_{q=0}\ne0.$ На самом деле, можно считать функцию $g$ линейной, т.к. при исследовании малых колебаний рассматривается не сами связи, а их линейные приближения.

Обобщённые координаты можно ввести так, что $g(q)=q_n$, т.~е. новая связь задаётся уравнением $q_n=0.$ Пусть $\mu_1\leqslant\ldots\leqslant \mu_{n-1}$ --- собственные значения новой системы.

\begin{theorem} 
При наложении связи новые собственные значения чередуются со старыми: $$\lambda_1\leqslant\mu_1\leqslant\lambda_2\leqslant\ldots\leqslant\mu_{n-1}\leqslant\lambda_n.$$
\end{theorem}

Доказательство основывается на применении теоремы Куранта.

Пусть $A$ --- матрица кинетической энергии: $A=a_{ij}(0),$ а $B$ --- матрица Гессе потенциальной энергии в нуле: $B=\frac{\partial^2 U(0)}{\partial q_i\partial q_j}.$ Числа $\lambda_1,\ldots,\lambda_n$ --- это собственные числа пары форм с матрицами $A$ и $B.$ В нормальных координатах имеем $(q,Bq)=\lambda_1q_1^2+\ldots+\lambda_nq_n^2$ на эллипсоиде $\{q~|~(q,Aq)=1\}.$

Рассмотрим $(n-1)$"=мерное подпространство, задаваемое уравнением $q_n=0$. Оно соответствует системе с новой связью. Обозначим через $\tilde L_k$ множество $k$"=мерных подпространств, содержащихся в нём. Очевидно, $\tilde L_k\subset L_k.$
Из первого равенства в теореме Куранта
$$\mu_k=\min_{W\in \tilde L_{k}}\max_{\substack{q\in W\\\|q\|=1}}(q,Bq)\geqslant\min_{W\in L_{k}}\max_{\substack{q\in W\\\|q\|=1}}(q,Bq)=\lambda_k,$$
а из второго
$$\mu_k=\max_{W\in\tilde L_{n-k}}\min_{\substack{q\in W\\\|q\|=1}}(q,Bq)\leqslant\max_{W\in L_{n-k+1}}\min_{\substack{q\in W\\\|q\|=1}}(q,Bq)=\lambda_{k+1}.$$
\qed

Собственные значения являются квадратами частот малых колебаний систем. Поэтому теорему 2 можно переформулировать следующим образом:
\slshape при наложении связи частоты колебаний новой системы чередуются с частотами старой.\upshape
\begin{thebibliography}{3}
\bibitem{Prasolov}
В. В. Прасолов.	Задачи и теоремы линейной алгебры, М.: МЦНМО, 2016
\bibitem{Bolotin}
С. В. Болотин и др. Теоретическая механика, М.: Академия, 2010
\end{thebibliography}
\end{document}
