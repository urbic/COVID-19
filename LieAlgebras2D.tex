\documentclass[a4paper,11pt]{article}
\usepackage{luatex85}
\usepackage[babelshorthands]{polyglossia}
\usepackage[fleqn,reqno]{amsmath}
\usepackage{amsthm}
\usepackage[math-style=ISO,bold-style=upright,partial=italic]{unicode-math}
\usepackage{microtype}
\usepackage[width=16cm,height=24cm]{geometry}
\usepackage[russian]{hyperref}
\usepackage{mathtools}

\setmainlanguage{russian}

\setmainfont{Cambria}
\setsansfont{Calibri}
\setmonofont{Source Code Pro}[Scale=MatchLowercase]
\setmonofont{Source Code Pro}[Scale=MatchLowercase]
\setmathfont{Cambria Math}[sans-style=literal]
\setmathfont{XITS Math}[range={\mathscr}]

\theoremstyle{definition}
\newtheorem{theorem}{Теорема}
\newtheorem*{theorem*}{Теорема}
\makeatletter

\allowdisplaybreaks[4]

\def\[#1\]{\begin{align*}#1\end{align*}}
\newcommand\eqtag[1]{\refstepcounter{equation}\tag{\theequation}\label{#1}}

%\newcommand\slashfrac[2]{{#1\fracslash#2}}
\newcommand\slashfrac[2]{{#1/#2}}


\begin{document}
	

\title{Классификация двумерных алгебр Ли}
\author{С.~Осинский}
\date{12.05.2020}
\maketitle

Пусть $\mathfrak g$~— произвольная комплексная(или вещественная) двумерная
алгебра Ли. Попытаемся определить общий вид как её самой, так и соответствующей
односвязной группы Ли. Для этого, для начала выберем в ней некоторый базис
$\{\mathbf{v,w}\}$. Тогда для произвольных элементов алгебры $\mathbf X$
и~$\mathbf Y$, используя свойства скобки, имеем:
	\[
	\mathbf X = a\mathbf v + b\mathbf w,
	\quad
	\mathbf Y=c\mathbf v+b\mathbf w,
	\quad
	a,b,c,d\in\BbbC.
	\]
Рассмотрим произвольную скобку:
	\[
	&[\mathbf X,\mathbf Y]=[a\mathbf v+b\mathbf w,c\mathbf v+d\mathbf w]
		=a[\mathbf v,c\mathbf v+d\mathbf w]+b[\mathbf w,c\mathbf v+d\mathbf w]={}\\
	&\qquad
		=ac[\mathbf v,\mathbf v]+ad[\mathbf v,\mathbf w]
		+bc[\mathbf w,\mathbf v]+bd[\mathbf w,\mathbf w]
		=(ad-bc)[\mathbf v,\mathbf w].
	\]
Получаем, что любая скобка пропорциональна скобке базисных элементов. Если
$[\mathbf v,\mathbf w]=\symbf0$, то нулю равна любая скобка и получается, что
наша алгебра абелева. Соответствующей односвязной группой Ли очевидно является
$\BbbC^2$ (или $\BbbR^2$). Пусть теперь $[\mathbf v,\mathbf w]\ne\mathbf0$.
Обозначим $\mathbf w_1=[\mathbf v,\mathbf w]$ Имеем:
	\[
	\forall\mathbf x,\mathbf y\in\mathfrak{g}:[\mathbf x,\mathbf y]
		=\lambda\mathbf w_1.
	\]
В~частности, для любого $\mathbf v_1$ такого, что $\mathbf v_1$ и $\mathbf w_1$
линейно независимы получаем:
	\[
	[\mathbf v_1,\mathbf w_1]=\lambda_0\mathbf w_1.
	\]
Заменяя $\mathbf v_1\to\frac1{\lambda_0}\mathbf v_1$ получаем базис $\{\mathbf
v_1,\mathbf w_1\}$ в~$\mathfrak g$, для которого выполняется $[\mathbf
v_1,\mathbf w_1]=\mathbf w_1$. Это выражение полностью определяет алгебру Ли,
и~мы получаем общий вид с~точностью до изоморфизма двумерной неабелевой алгебры
Ли. Теперь хочется получить его матричное представление. Рассмотрим
отображение:
	\[
	\phi\colon\mathfrak g\to\mathfrak{gl}(2,\BbbC),
	\quad
	\phi\colon a\mathbf v_1+b\mathbf w_1\mapsto
		\begin{pmatrix}
			a&b\\
			0&0
		\end{pmatrix}.
	\]
Покажем, что это отображение является изоморфизмом алгебр Ли. Для начала
проверим, что оно является гомоморфизмом алгебр Ли:
	\[
	\phi([a\mathbf v_1+b\mathbf w_1,c\mathbf v_1+d\mathbf w_1])=\phi((ad-bc)\mathbf w_1)
		=\begin{pmatrix}
			0&ad-bc\\
			0&0
		\end{pmatrix}.
	\]
В~то~же время:
	\[
	[\phi(a\mathbf v_1+b\mathbf w_1),\phi(c\mathbf v_1+d\mathbf w_1)]
		=\begin{pmatrix}
			a&b\\
			0&0
		\end{pmatrix}
		\begin{pmatrix}
			c&d\\
			0&0
		\end{pmatrix}
		-\begin{pmatrix}
			c&d\\
			0&0
		\end{pmatrix}
		\begin{pmatrix}
			a&b\\
			0&0
		\end{pmatrix}
		=\begin{pmatrix}
			0&ad-bc\\
			0&0
		\end{pmatrix}.
	\]
Кроме того, данное отображение, очевидно, инъективно (равенство элементов
матриц влечет равенство коэффициентов и соответственно равенство элементов
алгебры). Получили, что $\mathfrak g\simeq\phi(\mathfrak g)$, где:
	\[
	\phi(\mathfrak g)=\left\{
		\begin{psmallmatrix}
			a&b\\
			0&0
		\end{psmallmatrix}
		\in\mathfrak{gl}(2,\BbbC)\colon a,b\in\BbbC
		\right\}.
	\]
Далее нам потребуются некоторые дополнительные сведения из
книги~\cite{bib:vinberg}).

\begin{theorem}
Экспоненциальное отображение $\exp\colon\mathfrak g\to G$ диффеоморфно
отображает некоторую окрестность нуля в~алгебре $\mathfrak g$ в~некоторую
окрестность единицы группы $G$.
\end{theorem}

Таким образом, чтобы попасть в~группу Ли соответствующую нашей алгебре нам
нужно применить экспоненциальное отображение, в~данном случае совпадающее
с~классической матричной экспонентой (опять~же см.~\cite{bib:vinberg}).
Вычислим матричную экспоненту, пусть
$A=\begin{psmallmatrix}a&b\\0&0\end{psmallmatrix}$, найдем общий вид степени:
	\[
	A^n=\begin{pmatrix}
		a^n&a^{n-1}b\\
		0&0
		\end{pmatrix}.
	\]
Тогда имеем: 
	\[
	\exp(A)=\begin{pmatrix}
		e^a&\frac{(e^{a}-1)b}{a}\\
		0&1
	\end{pmatrix}.
	\]
Соответственно, попадая в группу Ли, мы получаем:
	\[
	G_0=\left\{
		\begin{psmallmatrix}
			a&b\\
			0&1
		\end{psmallmatrix}
		\colon a\ne0
		\right\}
		\subset GL_2(\BbbC).
	\]
Это одномерная аффинная группа. Мы хотим найти односвязную группу Ли, алгеброй
которой является двумерная неабелева алгебра Ли, для этого нам нужно найти
односвязное накрытие данной группы (их алгебры Ли совпадают,
см.~\cite{bib:vinberg}). Топологически эта группа устроена как
$\BbbC\times\BbbC^*$, для нахождения односвязного накрытия запишем ее
произвольный элемент в виде:
	\[
	\begin{pmatrix}
		e^t&s\\
		0&1
	\end{pmatrix}.
	\]
Произведение двух таких матриц имеет вид:
	\[
	\begin{pmatrix}
		e^t&s\\
		0&1
	\end{pmatrix}
	\begin{pmatrix}
		e^{t_1}&s_1\\
		0&1
	\end{pmatrix}
	=\begin{pmatrix}
		e^{t+t_1}&s+e^{t}s_1\\
		0&1
	\end{pmatrix}.
	\]
Поэтому можно рассматривать универсальное накрытие $G$ как группу пар
$(t,s)\in\BbbC\times\BbbC$ c~групповым законом
$(t,s)\cdot(t_1,s_1)=(t+t_1,s+e^ts_1)$. Накрывающий гомоморфизм
$f\colon\BbbC\times\BbbC\to\BbbC^*\times\BbbC$ по сути устроен так:
	\[
	(t,s)\mapsto(e^t,s),
	\]
из чего вытекает, что $\ker f=\{(2\pi in,0)\}\simeq\BbbZ$, $G_0\simeq
G/\ker{f}$ (cм.~\cite{bib:vinberg}). В~вещественном случае, пойдя по тому~же
пути, мы сразу после экспоненцирования попадаем в односвязную группу Ли, в~чем
легко убедиться, заметив, что соответствующее ядро накрывающего гомоморфизма
нулевое. Стоит также заметить, что все связные группы Ли с~такой алгеброй будут
факторами односвязной группы Ли по ее нормальным дискретным подгруппам. Таким
образом, в~вещественном случае можно довольно легко классифицировать
с~точностью до изоморфизма все связные двумерные группы Ли.

\begin{thebibliography}{9}
\bibitem{bib:vinberg} Э.~Б.~Винберг, А.~Л.~Онищик, Основы теории групп Ли.
\end{thebibliography}
\end{document}
